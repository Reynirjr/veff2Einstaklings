\documentclass{article}

% --- Packages og uppsetning --- 
\usepackage[top=0.9in, bottom=1in, left=1.5in, right=1.5in]{geometry}
\usepackage[icelandic]{babel}
\usepackage[T1]{fontenc}
\usepackage[sc]{mathpazo}
\usepackage[parfill]{parskip}
\usepackage{enumitem}
\usepackage{xcolor}
\usepackage{graphicx}
\usepackage{listingsutf8}

\setcounter{secnumdepth}{-1} 
\pagenumbering{gobble}

\title{Tónleikarnir – Einstaklingsverkefni (Fyrri skil)}
\author{(brj46)}
\date{Vor 2025}

\begin{document}

\maketitle

\vspace{2em}

\begin{center}
    \includegraphics[width=0.8\textwidth]{tonleikar.png}
\end{center}

\newpage

%--- 1. Inngangur ---
\section{Inngangur}
Markmið verkefnisins er að útbúa vefsíðu, kallaða \textbf{Tónleikarnir}, þar sem notendur geta:
\begin{itemize}
    \item Skráð sig inn og út (notendaumsjón).
    \item Myndað hópa eða gengið í hópa.
    \item Deilt YouTube-hlekkjum fyrir lag sem passar við gefið þema.
    \item Kosið um það lag sem þeim finnst best passa þemað. 
\end{itemize}

Ég fékk þessa hugmynd frá leik sem við spilum í vinnunni, þar sem hver og einn setur YouTube-lag sem hann telur passa við ákveðið þema, og svo kjósum við uppáhalds lagið okkar sem okkur finnst best passa við þemað.

%--- 2. Valin efnistök / Skilyrði ---
\section{Valin efnistök og skilyrði}
Verkefnið mun uppfylla a.m.k. þrjú af eftirtöldum atriðum:

\begin{enumerate}[label=\alph*.]
    \item \textbf{Bakendi:}  
        \begin{itemize}
            \item Node.js eða sambærilegt til að útfæra bakenda og notendaumsjón.
            \item Stjórnun á hópum, innskráningu/útskráningu, og atkvæðagreiðslu.
        \end{itemize}
    \item \textbf{Vefþjónusta (REST API):}  
        \begin{itemize}
            \item API sem sinnir beiðnum um að skrá nýja notendur, búa til hópa, setja inn lög og skila atkvæðum.
        \end{itemize}
    \item \textbf{Framendi:}  
        \begin{itemize}
            \item Síða þar sem einstaklingar geta skráð sig inn, valið hópa, sent inn hlekki að lagi, spilað YouTube-vídeó og kosið.
            \item framsetning með CSS (og smá css animations).
        \end{itemize}
\end{enumerate}

\section{Tæknin sem verður notuð}
\begin{itemize}
    \item \textbf{Bakendi:} Node.js með Express.js. 
    \item \textbf{Gagnagrunnur:} Einfaldur gagnagrunnur sennilega með postgreSQL. 
    \item \textbf{Framendi:}  
          \begin{itemize}
            \item Hannað með Figma.
            \item HTML og CSS og mögulega smá JavaScript.
            \item \textbf{YouTube Embed}: Sýna myndbönd í sennilega með \texttt{iframe} svo notendur fari ekki af síðunni til að horfa.
          \end{itemize}
\end{itemize}

%--- 4. Verkplan / Tímalína ---
\section{Verkplan og Tímalína (vikur 6--13)}
\begin{enumerate}[label=\arabic*.]
    \item \textbf{Vika 6--7:}  
    \begin{itemize}
        \item Grunnhönnun: Hvernig notendur skrá sig, hópamyndun og flæði atkvæðagreiðslu og figma hönnunarskjöl.
    \end{itemize}
    \item \textbf{Vika 8--9:}  
    \begin{itemize}
        \item Útfæra bakenda: Notendaumsjón, inn- og útskráningu, geymslu á lögum og atkvæðum í gagnagrunni.
        \item Setja upp einfalt REST API (t.d. \texttt{/login}, \texttt{/groups}, \texttt{/songs}).
    \end{itemize}
    \item \textbf{Vika 10--11:}  
    \begin{itemize}
        \item Tenging framenda og bakenda: klára búa til síðuna sjálfa (HTML/CSS/JS).
        \item Bæta við YouTube embed svæðum til að spila lög beint á síðunni.
    \end{itemize}
    \item \textbf{Vika 12:}  
    \begin{itemize}
        \item Prófanir og villuleit.
        \item Uppsetning í hýsingu ef tími gefst.
    \end{itemize}
    \item \textbf{Vika 13:}  
    \begin{itemize}
        \item Fínpússun og lokaskýrslugerð.
        \item Undirbúningur fyrir kynningu.
    \end{itemize}
\end{enumerate}

%--- 5. Matskvarði ---
\section{Matskvarði}
\begin{itemize}
    \item \textbf{Bakendi og notendaumsjón (35\%):}  
    Hvernig tekst til með innskráningu, útskráningu, hópastjórnun og atkvæði.
    \item \textbf{Vefþjónusta (20\%):}  
    Skýr og vel skilgreind REST API sem nær yfir helstu gögn (notendur, hópa, lög, atkvæði).
    \item \textbf{Framendi (35\%):}  
    Þægilegt notendaviðmót sem sýnir YouTube video, tekur við gögnum og auðveldar kosningu.
    \item \textbf{Skýrsla og hönnunarskjöl (10\%):}  
    Skýrsla um framgang verkefnis + grunnteikningar.
\end{itemize}

%--- 6. Kynning ---
\section{Kynning}
Ég ætla að kynna verkefnið!

%--- 7. Lokaorð / næstu skref ---
\section{Lokaorð}
Þetta verkefni einbeitir sér að einföldum en skýrum þáttum: notendaumsjón, hópamyndun, lagaskil með YouTube embed og kosningakeppni. 

\end{document}
